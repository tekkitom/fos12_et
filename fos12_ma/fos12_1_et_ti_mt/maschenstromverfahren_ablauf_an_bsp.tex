\documentclass[aspectratio=169, ignorenonframetext]{beamer}
\usepackage{pgf}

\mode<article>{
  \usepackage{fullpage}

}

\mode<presentation>
{
%   \usetheme{Madrid}
%   \useinnertheme{circles}
  \setbeamertemplate{navigation symbols}{}
%   \usetheme[hideothersubsections, width=2.4cm]{Hannover}
%   \usetheme{Antibes}
%   \usetheme{Montpellier}
  \usetheme{Singapore}
%   \usecolortheme{seahorse}
   \setbeamertemplate{footline}
   {%
%      \begin{beamercolorbox}{section in head/foot}
%       \usebeamercolor{bg}
       \hskip 1em \footnotesize \insertframenumber{} / \inserttotalframenumber%\hskip 41em \includegraphics[height = .5cm]{../../../../bilder/cc_by-nc_eu.png}
% % cc_by-nc_eu.png: 403x141 px, 72dpi, 14.22x4.97 cm, bb=0 0 403 141
%
% % bwslogo_3.png: 476x392 px, 300dpi, 4.03x3.32 cm, bb=0 0 114 94
%
%       %\hskip 5em
%       %       \input{../bilder/cc_by.png}
%       %\includegraphics[height=.5cm]{../../../../bilder/bwslogo_3.png}
%      \end{beamercolorbox}%
   }
  \usepackage{beamerfoils}
}
\usepackage[german]{babel}
\usepackage[utf8]{inputenc}
\usepackage{times}
\usepackage[T1]{fontenc}
\usepackage{eurosym}
\usepackage{graphicx}
\usepackage{amsmath}
\usepackage[siunitx,european]{circuitikz}
\usepackage{ulem}
\usepackage{listings}
%
\lstset{numbers=left, numberstyle=\tiny, stepnumber=2, numbersep=5pt, language = C++, alsolanguage=XML}
% \MyLogo{\includegraphics[height=1cm]{../../../../bilder/bwslogo_3.png}}
% % \includegraphics{../../bilder/bwslogo_3.png}
% % bwslogo_3.png: 476x392 px, 300dpi, 4.03x3.32 cm, bb=
%
\only<article>{
  \usepackage[colorlinks=true,linkcolor=blue,filecolor=magenta,urlcolor=cyan]{hyperref}
}

\only<presentation>{
  \usepackage{hyperref}
}


\title{Maschenstromverfahren / Kreisstromverfahren}
\subtitle{Ein Beispiel}
% \date{V 0.2.0 - im Aufbau\\ Stand: \today}%\\

\institute[BWS Hofheim]{Brühlwiesenschule, Hofheim}
\author{Thomas Maul}

\titlegraphic{Für eigene Teile gilt: \includegraphics[height=1cm]{cc_by-nc_eu.png}}

\begin{document}
%   \only<article>{
%     \maketitle
%     \tableofcontents
%     \clearpage
%   }
\begin{frame}<beamer>
  \titlepage
  % \hyperlink{Teil_2}{\beamerbutton{Go part 2}}
\end{frame}

\begin{frame}{Aufgabe und Quelle}

 Die Aufgabe stammt von Wolfgang Kippels \\
 https://dk4ek.de/doku.php/elektronik

 Hier Aufgabe 16. Auf der Webseite ist ein Script mit Aufgaben und ein weiteres mit ausführlichen Lösungen.
\end{frame}

\begin{frame}{Aufgabenstellung}


\begin{figure}
 \includegraphics{aufgabe_16/aufgabe16_lsg_baum_maschen_0.png}
 % aufgabe16_lsg_baum_maschen_1.png: 1063x400 px, 300dpi, 9.00x3.39 cm, bb=0 0 255 96
 \label{abb:Aufg16_ausgang}
\end{figure}
\end{frame}

\begin{frame}{Baum}
\begin{figure}
 \includegraphics{aufgabe_16/aufgabe16_lsg_baum_maschen_1.png}
 % aufgabe16_lsg_baum_maschen_1.png: 1063x400 px, 300dpi, 9.00x3.39 cm, bb=0 0 255 96
 \label{abb:Aufg16_BaumMa}
\end{figure}
\end{frame}

\begin{frame}{Baum und Maschen}
\begin{figure}
 \includegraphics{aufgabe_16/aufgabe16_lsg_baum_maschen_2.png}
 % aufgabe16_lsg_baum_maschen_1.png: 1063x400 px, 300dpi, 9.00x3.39 cm, bb=0 0 255 96
 \label{abb:Aufg16_MaschenMa}
\end{figure}
\end{frame}

\begin{frame}{LGS aus Schaltung}
\begin{figure}
 \includegraphics{aufgabe_16/aufgabe16_lsg_baum_maschen_2.png}
 % aufgabe16_lsg_baum_maschen_1.png: 1063x400 px, 300dpi, 9.00x3.39 cm, bb=0 0 255 96
 \label{abb:Aufg16_MaschenMa2}
\end{figure}

\[ \left(
\begin{matrix}
R_1+R_2 & -R_2 & -R_2\\
-R_2 & R_2 + R_3 + R_4 & R_2+R_3\\
-R_2 & R_2 + R_3 & R_2+R_3+R_5
\end{matrix}
\right)
\cdot
\left (
\begin{matrix}
I_{M1}\\
I_{M2}\\
I_{M3}
\end{matrix}
\right )
=
\left (
\begin{matrix}
U_0\\
0\\
0
\end{matrix}
\right )
\]

\end{frame}

\begin{frame}{LGS aus Schaltung}

\[ \left(
\begin{matrix}
5\,k\Omega & -1\,k\Omega & -1\,k\Omega\\
-1\,k\Omega & 13,1\ k\Omega & 4,1 k\Omega\\
-1\,k\Omega & 4,1 k\Omega & 5,1\ k\Omega
\end{matrix}
\right)
\cdot
\left (
\begin{matrix}
I_{M1}\\
I_{M2}\\
I_{M3}
\end{matrix}
\right )
=
\left (
\begin{matrix}
12\,V\\
0\\
0
\end{matrix}
\right )
\]

\[
U_0 = 12\,V, R_1 = 4\,k\Omega, R_2 = 1\,k\Omega, R_3 = 3,1\ k\Omega, R_4 = 9\,k\Omega R_5 = 1\,k\Omega
\]

\end{frame}
\begin{frame}{LGS aus Schaltung}

$ \left(
\begin{array}{ccc|c}
5\,k\Omega & -1\,k\Omega & -1\,k\Omega & 12\,V\\
-1\,k\Omega & 13,1\ k\Omega & 4,1 k\Omega & 0\\
-1\,k\Omega & 4,1 k\Omega & 5,1\ k\Omega & 0
\end{array}
\right)
$
% \begin{tabular}{ccc|c}
%  &&&\\
%  &&&\\
%  &&&\\
% \end{tabular}
\end{frame}

\label{lastpage}
\end{document}


