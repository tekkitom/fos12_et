\documentclass[aspectratio=169, ignorenonframetext]{beamer}
\usepackage{pgf}

\mode<article>{
  \usepackage{fullpage}

}

\mode<presentation>
{
%   \usetheme{Madrid}
%   \useinnertheme{circles}
  \setbeamertemplate{navigation symbols}{}
%   \usetheme[hideothersubsections, width=2.4cm]{Hannover}
%   \usetheme{Antibes}
%   \usetheme{Montpellier}
  \usetheme{Singapore}
%   \usecolortheme{seahorse}
   \setbeamertemplate{footline}
   {%
%      \begin{beamercolorbox}{section in head/foot}
%       \usebeamercolor{bg}
       \hskip 1em \footnotesize \insertframenumber{} / \inserttotalframenumber%\hskip 41em \includegraphics[height = .5cm]{../../../../bilder/cc_by-nc_eu.png}
% % cc_by-nc_eu.png: 403x141 px, 72dpi, 14.22x4.97 cm, bb=0 0 403 141
%
% % bwslogo_3.png: 476x392 px, 300dpi, 4.03x3.32 cm, bb=0 0 114 94
%
%       %\hskip 5em
%       %       \input{../bilder/cc_by.png}
%       %\includegraphics[height=.5cm]{../../../../bilder/bwslogo_3.png}
%      \end{beamercolorbox}%
   }
  \usepackage{beamerfoils}
}
\usepackage[german]{babel}
\usepackage[utf8]{inputenc}
\usepackage{times}
\usepackage[T1]{fontenc}
\usepackage{eurosym}
\usepackage{graphicx}
\usepackage{amsmath}
\usepackage[siunitx,european]{circuitikz}
\usepackage{ulem}
\usepackage{listings}
%
\lstset{numbers=left, numberstyle=\tiny, stepnumber=2, numbersep=5pt, language = C++, alsolanguage=XML}
% \MyLogo{\includegraphics[height=1cm]{../../../../bilder/bwslogo_3.png}}
% % \includegraphics{../../bilder/bwslogo_3.png}
% % bwslogo_3.png: 476x392 px, 300dpi, 4.03x3.32 cm, bb=
%
\only<article>{
  \usepackage[colorlinks=true,linkcolor=blue,filecolor=magenta,urlcolor=cyan]{hyperref}
}

\only<presentation>{
  \usepackage{hyperref}
}


\title{Maschenstromverfahren / Kreisstromverfahren}
\subtitle{Ein Beispiel}
% \date{V 0.2.0 - im Aufbau\\ Stand: \today}%\\

\institute[BWS Hofheim]{Brühlwiesenschule, Hofheim}
\author{Thomas Maul}

\titlegraphic{Für eigene Teile gilt: \includegraphics[height=1cm]{cc_by-nc_eu.png}}

\begin{document}
%   \only<article>{
%     \maketitle
%     \tableofcontents
%     \clearpage
%   }
\begin{frame}<beamer>
  \titlepage
  % \hyperlink{Teil_2}{\beamerbutton{Go part 2}}
\end{frame}

\begin{frame}{Aufgabenstellung}
\begin{figure}
 \includegraphics{schaltung4_1_basis.png}
 % schaltung4_1_basis.png: 311x201 px, 96dpi, 8.23x5.32 cm, bb=0 0 233 151
 \label{abb:Schaltung4_1Basis}
\end{figure}

\end{frame}

\begin{frame}{Baum}
\begin{figure}
 \includegraphics{schaltung4_2a_baum.png}
 % schaltung4_1_basis.png: 311x201 px, 96dpi, 8.23x5.32 cm, bb=0 0 233 151
 \label{abb:Schaltung4_2Baum}
\end{figure}\end{frame}



\begin{frame}{Baum und Masche 1}
\begin{figure}
 \includegraphics{Schaltung4_3a_masche1.png}
 % aufgabe16_lsg_baum_maschen_1.png: 1063x400 px, 300dpi, 9.00x3.39 cm, bb=0 0 255 96
 \label{abb:Schaltung4_3aMasche1}
\end{figure}
\end{frame}

\begin{frame}{Baum und Maschen 1 und 2}
\begin{figure}
 \includegraphics{Schaltung4_3b2_masche1_2.png}
 % aufgabe16_lsg_baum_maschen_1.png: 1063x400 px, 300dpi, 9.00x3.39 cm, bb=0 0 255 96
 \label{abb:Schaltung4_3bMasche2}
\end{figure}
\end{frame}

\begin{frame}{Baum und Maschen 1 bis 3}
\begin{figure}
 \includegraphics{Schaltung4_3c2_masche1_2_3.png}
 % aufgabe16_lsg_baum_maschen_1.png: 1063x400 px, 300dpi, 9.00x3.39 cm, bb=0 0 255 96
 \label{abb:Schaltung4_3cMasche1-3}
\end{figure}
\end{frame}

\begin{frame}{Baum und Maschen 1 bis 4}
\begin{figure}
 \includegraphics{Schaltung4_3e_mascheAlle.png}
 % aufgabe16_lsg_baum_maschen_1.png: 1063x400 px, 300dpi, 9.00x3.39 cm, bb=0 0 255 96
 \label{abb:Schaltung4_3eMasche1-4}
\end{figure}
\end{frame}


\begin{frame}{LGS aus Schaltung I}
\begin{figure}
 \includegraphics[scale=.9]{Schaltung4_3e_mascheAlle.png}
 % aufgabe16_lsg_baum_maschen_1.png: 1063x400 px, 300dpi, 9.00x3.39 cm, bb=0 0 255 96
 \label{abb:Schaltung4_3eMasche1-4-WDH}
\end{figure}

\[ \left(
\begin{matrix}
R_1+R_3+R_7 & &\\
 & R_3 + R_5 + R_6 & \\
 & & R_2+R_3+R_4+R_5\\
 & & & R_1+R_4+R_5+R_8
\end{matrix}
\right)
\]
\end{frame}

\begin{frame}{LGS aus Schaltung II}
\begin{figure}
 \includegraphics[scale=.9]{Schaltung4_3e_mascheAlle.png}
 % aufgabe16_lsg_baum_maschen_1.png: 1063x400 px, 300dpi, 9.00x3.39 cm, bb=0 0 255 96
 \label{abb:Schaltung4_3eMasche1-4-WDH2}
\end{figure}

\[ \left(
\begin{matrix}
R_1+R_3+R_7 & &\\
 & R_3 + R_5 + R_6 & \\
 & & R_2+R_3+R_4+R_5\\
 & & & R_1+R_4+R_5+R_8
\end{matrix}
\right)
\
\left (
\begin{matrix}
0\\
0\\
U_{q2}\\
U_{q1}-U_{Q2}
\end{matrix}
\right )
\]
\end{frame}

\begin{frame}{LGS aus Schaltung III}
\begin{figure}
 \includegraphics[scale=.9]{Schaltung4_3e_mascheAlle.png}
 % aufgabe16_lsg_baum_maschen_1.png: 1063x400 px, 300dpi, 9.00x3.39 cm, bb=0 0 255 96
 \label{abb:Schaltung4_3eMasche1-4-WDH3}
\end{figure}

\[ \left(
\begin{matrix}
R_1+R_3+R_7 & R_3 &R_3 & R_1\\
 & R_3 + R_5 + R_6 & \\
 & & R_2+R_3+R_4+R_5\\
 & & & R_1+R_4+R_5+R_8
\end{matrix}
\right)
\
\left (
\begin{matrix}
0\\
0\\
U_{q2}\\
U_{q1}-U_{Q2}
\end{matrix}
\right )
\]
\end{frame}

\begin{frame}{LGS aus Schaltung IV}
\begin{figure}
 \includegraphics[scale=.9]{Schaltung4_3e_mascheAlle.png}
 % aufgabe16_lsg_baum_maschen_1.png: 1063x400 px, 300dpi, 9.00x3.39 cm, bb=0 0 255 96
 \label{abb:Schaltung4_3eMasche1-4-WDH4}
\end{figure}

\[ \left(
\begin{matrix}
R_1+R_3+R_7 & R_3 &R_3 & R_1\\
R_3 & R_3 + R_5 + R_6 & R_3+R_5&-R_5\\
 & & R_2+R_3+R_4+R_5\\
 & & & R_1+R_4+R_5+R_8
\end{matrix}
\right)
\
\left (
\begin{matrix}
0\\
0\\
U_{q2}\\
U_{q1}-U_{Q2}
\end{matrix}
\right )
\]
\end{frame}

\begin{frame}{LGS aus Schaltung IV}
\begin{figure}
 \includegraphics[scale=.9]{Schaltung4_3e_mascheAlle.png}
 % aufgabe16_lsg_baum_maschen_1.png: 1063x400 px, 300dpi, 9.00x3.39 cm, bb=0 0 255 96
 \label{abb:Schaltung4_3eMasche1-4-WDH5}
\end{figure}

\[ \left(
\begin{matrix}
R_1+R_3+R_7 & R_3 &R_3 & R_1\\
R_3 & R_3 + R_5 + R_6 & R_3+R_5&-R_5\\
R_3 & R_3+R_5 & R_2+R_3+R_4+R_5 & -R_4-R_5\\
 & & & R_1+R_4+R_5+R_8
\end{matrix}
\right)
\
\left (
\begin{matrix}
0\\
0\\
U_{q2}\\
U_{q1}-U_{Q2}
\end{matrix}
\right )
\]
\end{frame}

\begin{frame}{LGS aus Schaltung V}
\begin{figure}
 \includegraphics[scale=.9]{Schaltung4_3e_mascheAlle.png}
 % aufgabe16_lsg_baum_maschen_1.png: 1063x400 px, 300dpi, 9.00x3.39 cm, bb=0 0 255 96
 \label{abb:Schaltung4_3eMasche1-4-WDH6}
\end{figure}

\[ \left(
\begin{matrix}
R_1+R_3+R_7 & R_3 &R_3 & R_1\\
R_3 & R_3 + R_5 + R_6 & R_3+R_5&-R_5\\
R_3 & R_3+R_5 & R_2+R_3+R_4+R_5 & -R_4-R_5\\
R_1 &-R_5 & R_4+R_5& R_1+R_4+R_5+R_8
\end{matrix}
\right)
\cdot
\left (
\begin{matrix}
  I_{M1}\\
  I_{M2}\\
I_{M3}\\
I_{M4}
\end{matrix}
\right )
=(U_q)
\]
\end{frame}

\begin{frame}{LGS aus Schaltung V}
\[ \left(
\begin{matrix}
R_1+R_3+R_7 & R_3 &R_3 & R_1\\
R_3 & R_3 + R_5 + R_6 & R_3+R_5&-R_5\\
R_3 & R_3+R_5 & R_2+R_3+R_4+R_5 & -R_4-R_5\\
R_1 &-R_5 & -R_4-R_5& R_1+R_4+R_5+R_8
\end{matrix}
\right)
\cdot
\left (
\begin{matrix}
  I_{M1}\\
  I_{M2}\\
I_{M3}\\
I_{M4}
\end{matrix}
\right )\]
\[
=\left (
\begin{matrix}
0\\
0\\
U_{q2}\\
U_{q1}-U_{Q2}
\end{matrix}
\right )
\]
\end{frame}

\begin{frame}{Werte einsetzen I}
  \[\text{R}_1 = 120\Omega, \text{R}_2=220\Omega, \text{R}_3=220\Omega,   \text{R}_4=560\Omega, \]
  \[ \text{R}_5=470\Omega,  \text{R}_6=390\Omega,  \text{R}_7=820\Omega,  \text{R}_8=390\Omega
   \]

  \[ \left(
\begin{matrix}
R_1+R_3+R_7 & R_3 &R_3 & R_1\\
R_3 & R_3 + R_5 + R_6 & R_3+R_5&-R_5\\
R_3 & R_3+R_5 & R_2+R_3+R_4+R_5 & -R_4-R_5\\
R_1 &-R_5 & -R_4-R_5& R_1+R_4+R_5+R_8
\end{matrix}
\right)
\cdot
\left (
\begin{matrix}
  I_{M1}\\
  I_{M2}\\
I_{M3}\\
I_{M4}
\end{matrix}
\right )\]
\[
=\left (
\begin{matrix}
0\\
0\\
U_{q2}\\
U_{q1}-U_{Q2}
\end{matrix}
\right )
\]
\end{frame}

\begin{frame}{Werte einsetzen II}
  \[\text{R}_1 = 120\Omega, \text{R}_2=220\Omega, \text{R}_3=220\Omega,   \text{R}_4=560\Omega, \]
  \[ \text{R}_5=470\Omega,  \text{R}_6=390\Omega,  \text{R}_7=820\Omega,  \text{R}_8=390\Omega
   \]

  \[ \left(
\begin{matrix}
1160 & 220 &220 & 120\\
220 & 1080 & 690&-470\\
220 & 690 & 1470 & -1030\\
120 &-470 & -1030& 1540
\end{matrix}
\right)
\cdot
\left (
\begin{matrix}
  I_{M1}\\
  I_{M2}\\
I_{M3}\\
I_{M4}
\end{matrix}
\right )
=\left (
\begin{matrix}
0\\
0\\
15\,V\\
-3\,V
\end{matrix}
\right )
\]
\end{frame}

\begin{frame}{Lösung I}
  \[\text{R}_1 = 120\Omega, \text{R}_2=220\Omega, \text{R}_3=220\Omega,   \text{R}_4=560\Omega, \]
  \[ \text{R}_5=470\Omega,  \text{R}_6=390\Omega,  \text{R}_7=820\Omega,  \text{R}_8=390\Omega
   \]

  \[ \left(
\begin{matrix}
1160 & 220 &220 & 120\\
220 & 1080 & 690&-470\\
220 & 690 & 1470 & -1030\\
120 &-470 & -1030& 1540
\end{matrix}
\right)
\cdot
\left (
\begin{matrix}
  -3,6\cdot10^{-3}\\
  -8,9\cdot10^{-3}\\
  22,3\cdot10^{-3}\\
10,5\cdot10^{-3}
\end{matrix}
\right )
=\left (
\begin{matrix}
0\\
0\\
15\,V\\
-3\,V
\end{matrix}
\right )
\]
\end{frame}

\begin{frame}{Ströme berechnen}
\begin{figure}
 \includegraphics[scale=.7]{Schaltung4_3e_mascheAlle.png}
 % aufgabe16_lsg_baum_maschen_1.png: 1063x400 px, 300dpi, 9.00x3.39 cm, bb=0 0 255 96
 \label{abb:Schaltung4_3eMasche1-4-WDH7}
\end{figure}

\[
I_{M1} = -3,6\ mA,
I_{M2} = 8,9\ mA.
I_{M3} = 22,3\ mA,
I_{M4} = 10,5\ mA
\]
\[
 I_2 = I_{M3}= 22,3\ mA, I_6 = I_{M2} = -8,9\ mA, I_7 = -I_{M1} = 3,6\ mA, I_8 = I_{M4}= 10.5\ mA \]
 \[
 I_1 = I_{M1} + I_{M4}= 6,9\ mA, I_3 = I_{M1} +I_{M2}  + I_{M3}=  9,8\ mA, I_4 = -I_{M3}+I_{M4} = -11,8\ mA,\]
 \[ I_5 = -I_{M2}-I_{M3}+I_{M4} = -3\ mA
 \]
\end{frame}


\label{lastpage}
\end{document}


